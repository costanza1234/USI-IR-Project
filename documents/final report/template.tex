\documentclass[unicode,9pt,a4paper,oneside,numbers=endperiod,openany]{scrartcl}

\renewcommand{\thesubsection}{\arabic{subsection}}

\input{assignment.sty}
\usepackage{amssymb}
\begin{document}


\setassignment
\setduedate{Friday, 20 December 2024, 11:59 PM}

\serieheader{Information Retrieval}{2024}{\textbf{Student:} Costanza Rodiguez Gavazzi, Agnese Zamboni, Davide Frova}{}
\newline

\section{Overview}

\section{Frontend Design and Implementation}

\section{Data Collection and Processing}

\subsection{Global Giving}
Global Giving provides structured data on charitable organizations through its API in XML format. We utilized this API to collect comprehensive information about each organization, ensuring a reliable and standardized method of data retrieval.

\subsubsection{Data Retrieval}
To gather data, we accessed the Global Giving API and downloaded XML files containing details of various organizations. The use of their API eliminated the need for web scraping, allowing us to work directly with structured data. The XML files provided:
\begin{itemize}
\item Basic details such as the organization name and location.
\item Operational metrics, including active and total projects.
\item Mission statements and website URLs.
\item Themes representing the causes they work on.
\item Countries where the organizations operate.
\end{itemize}

\subsubsection{Data Processing}
After retrieving the data, we developed a parser using Python's \texttt{ElementTree} library to extract information from the XML format. To enhance the dataset:
\begin{itemize}
\item Geographical information was added using \texttt{pycountry} and \texttt{pycountry\_convert}, allowing us to classify organizations by continent based on their headquarters.
\item Field names were standardized to align with the data structure of Charity Navigator, ensuring consistency across platforms.
\item The final dataset was converted to JSON format for easier storage and readability.
\end{itemize}

\subsection{Charity Navigator}
Charity Navigator offers data through its GraphQL API. Unlike Global Giving, Charity Navigator does not provide organization logos, requiring additional processing to locate this information.

\subsubsection{Data Retrieval}
Using the GraphQL API, we tailored queries to retrieve:
\begin{itemize}
\item Organization details, including names and locations.
\item Ratings and operational metrics.
\item Mission statements and website URLs.
\end{itemize}
The GraphQL API's flexibility allowed us to avoid redundant data requests and efficiently handle nested data structures. 10,000 records were retrieved in batches of 10 to not flood the server.

\subsubsection{Data Processing}
The lack of logos in Charity Navigator's data required us to develop a custom solution for locating and extracting organization logos from their websites. This system:
\begin{itemize}
\item Parsed webpage structures using \texttt{BeautifulSoup} to identify elements marked as logos.
\item Checked metadata and special tags (e.g., \texttt{}) for logo information.
\item Scanned for images commonly used as logos, filtering out irrelevant elements like favicons or menu icons.
\end{itemize}
Standard Python libraries, including \texttt{requests}, \texttt{BeautifulSoup}, and \texttt{re}, facilitated these operations. Finally, the processed data was converted to JSON format to match the structure of Global Giving's dataset.

The Charity Navigator system enabled detailed data extraction while addressing the challenge of missing logos. The resulting dataset, enriched with logos and stored in JSON format, supports comprehensive analysis and cross-platform comparisons.


\section{Backend and Indexing/Retrieval}

\section{User Evaluation}

\section{Appendix}


\end{document}